\textbf{\title{\normalsize{Definiteness}}}
\section*{Syntax and Semantics}
\hspace{1em} If we compare the syntax of Kotlin to Java almost everything is entirely new. Instead of declaring a variable with its data type, like a lot of the strongly typed languages, Kotlin starts the declaration with either val or var. Val means that the value the variable takes on is immutable while var is mutable. One plain example in how things are declared is the declaration of primitive arrays, that is arrays that are not declared using the Array class: \textit{val x: IntArray = intArrayOf(1, 2, 3)}; which means the value x, which is immutable, is an integer array containing the values 1, 2, and 3. Values from this array can be called as expected, x[0]. Also all of the types (int, boolean, double, etc) are declared after the variable name. However, as stated previously there are times where we do not need to declare the type of the variable because the compiler can infer. 

\hspace{1em} Kotlin also uses the `?' operator to check if something will output a null, or simply to box variables into say the java.lang.Integer class: \textit{val bob?.department} or \textit{val a: Int? = 1} respectively. The first one checks to see if bob is in a department, if he is it will return the value of the department otherwise it will return null. For the second one it is checking to see if the following number can safely be boxed into the java.lang.Integer class. Another useful syntax is when an argument has to call upon a range of numbers, \textit{if(c in 0..9)}. When there's an argument that is searching through a range of numbers it simply means that if the c is in the range from zero to nine, then execute. The range can be used in any conditional statement. `!in' could also be used which is the negation of the `in.'

\hspace{1em}For many other structures, if-else, while, for, etc they are all written the same as in Java. Looking back at the example of the foreach loop \textit{for(c in str)}, this means precisely that for each element c, that is for each character, in the string variable str. So the loop will loop as long as a character remains in the string str. Since there are many primitive types there are also many ways to convert them, for example if an argument needed a long but instead the variable was an integer it could be changed with the following: \textit{x.toLong()} assuming that x is an integer. This can also be done with strings to characters or a numerical to a character. A function can be define with the keyword `fun.' It is much like a method in Java and it takes input quite the same way, for example \textit{fun printer(str: String)}. This function is named printer and take a string input. For anything that is syntactically the same in Kotlin as it is in Java, the semantics are also the same.